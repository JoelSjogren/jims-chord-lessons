\section{Lesson 4 - Fully Diminished Chords}
So far, we have looked at the structures of the I and II chords that come
from major keys, and the dominant chord. For a comparison of their
structures, I give you the following table, quoted from the end of the last
lesson:

-----------------------------------------
| -13  M13      |   M13    |  M13       |
|      P11  A11 |   P11    |  P11  A11  |
|  -9   M9   A9 |    M9    |   M9       |
|       -7      |    -7    |   M7       |
|       !6      |    !6    |   M6       |
|  d5   P5   A5 |    P5    |   P5 (A5)  |
|       P4      |    !4    |   P4       |
|       M3      |    -3    |   M3       |
|      Root     |   Root   |  Root      |
|---------------|----------|-------------
|        V      | II-Major | I-Major    |
-----------------------------------------

Notice that THE EXTENSIONS (9ths, 11ths, 13ths) DON'T TELL YOU MUCH in the way
of comparison! In fact, the unaltered extensions are !always! \emph{major} ninth,
\emph{perfect} eleventh and \emph{major} thirteenth. So how do we compare chords? The
fifths don't provide much information either, and the roots provide exactly no
information. In the main, the third and the seventh will provide the keys to
determine the family identity to which a given chord belongs.

>From the second article:

Note that all mention of intervals are intervals measured from the ROOT
of the chord. The same is true for my first article.

Recall also that the goal of this series is to impart a knowledge of
chords such as those you might see in a jazz fake book, for example

Dm11     G13     Cmaj.7

Specifically, the goal is to be able to spell a chord given its name and
to give the name of the chord given its spelling. For 3 examples, the
Dm11 above is spelled D, F, A, C, E and G; the G13 is G, B, D, F, A, C
and E; and the C Maj. 7 is C, E, G and B.

Also, the chords covered here are ones that you could strike all the
notes at the same time and hold them forever without being overly
dissonant. Notes of chords which do not fall into this category would
probably be considered non-harmonic or at least "un-hip".

(unquote)

>From the first article:

There are roughly 6 different chordal groupings you would need to worry about:
Tonic major chords, tonic minor, supertonic major, supertonic minor, dominant
and fully diminished. Each one of these has several possibilities for scales
to go with them (however, there are also non-scalar possibilities as well).

(unquote)

We now turn to the bridging group of chords. The bridging group consists of
fully diminished 7th chords. Bridging chords have NO extensions. Bridging
chords do not come from a key; they are, in this sense, "man-made".

All diminished chords, whether fully diminished or half, have a diminished
triad. To review: A diminished triad has a root, a minor third and a diminished
fifth.

 d5
 m3
root

For example C, Eb and Gb is a diminished triad.

Overall, there are two choices for the seventh: either minor or diminished.
So what does 'fully' and 'half' mean? What are the halves? Here's the answer:
The matter of full versus half involves the triad as one half and the seventh
as the other half. So: if the triad is diminished and the seventh is minor,
then only half of the chord is diminished, so we call this structure:

 -7
 d5
 m3
root

a half-diminished seventh chord. (On C, this is spelled C, Eb, Gb and Bb.)

On the other hand, if BOTH the seventh AND the triad are diminished, as in:

 d7
 d5
 m3
root

then the chord is a \emph{fully} diminished seventh chord.

In this lesson, we deal only with fully diminished seventh chords, leaving
the half-diminished seventh to be discussed later.

Before, I said that one of the chord groupings is not what is referred to
here as a family, but is just a group. This is because chords from this group
do not come from (or have any particular 'allegience' to) any particular key.
The bridging group (the subject of this lesson) is the chord group that is not
a family. Here's why:

If we look at an octave and look at all notes possible, we see that there
are twelve different notes. This does not include the top note, which is just
a repeat of the bottom. This is called a chromatic scale.

One possible chromatic scale (on C):

C    C\#   D    D\#   E    F    F\#   G    G\#   A    A\#   B    (C)

Since we have 12 different notes, it's possible to divide this octave equally.
The even divisions of 12 are 2, 3, 4 and 6.

For example, a whole-tone scale is formed by taking every other note of a
chromatic scale. A C whole-tone scale looks like this:

C         D         E         F\#        G\#        A\#        (C)
C    C\#   D    D\#   E    F    F\#   G    G\#   A    A\#   B    (C)

A chromatic scale is all half steps; a whole-tone scale is all \emph{whole} steps.
So a whole-tone scale is said to equally divide the octave. How many different
notes are there? Each tone is two half steps from the last, and there are 12
half steps in a chromatic scale. 12 divided by 2 is 6, so there are 6
(different) notes in a whole-tone scale. If there are 6 different notes in
a whole-tone sca;e. how many different whole-tone scales are there? 12 divided
by 6 is 2, so there are only two different whole-tone scales: on C, and on C\#.
Why not on D? Because a D whole-tone scale is exactly the same as a C whole-
tone scale.

3 goes into 12 as well. We can come up with a "scale" whose tones are three
half steps apart, and which equally divides the octave:

C              D\#             F\#             A              (C)
C    C\#   D    D\#   E    F    F\#   G    G\#   A    A\#   B    (C)

We can do the same for a "scale" whose tones are four half steps apart. Four
goes into 12 three times, so there will be three different notes:

C                   E                   G\#                  (C)
C    C\#   D    D\#   E    F    F\#   G    G\#   A    A\#   B    (C)

If there are two tones, they will be spaced \emph{six} half steps apart:

C                             F\#                            (C)
C    C\#   D    D\#   E    F    F\#   G    G\#   A    A\#   B    (C)

If we thought of some of these as chords, how would you tell for sure what the
root of the chord is? You can't tell for sure! The C-E-G\# happens to be an
augmented triad. It would sound EXACTLY the same as an E augmented triad! It
would sound EXACTLY the same as a G\# (or Ab) augmented triad! If there are no
chords before or after (one-chord jam), you can't tell (by listening) what key
it comes from or even what its root is! (Bass players: this gives you the POWER
to decide the root! Others: be NICE go your bass player! He/she has power!)

A fully diminished chord equally divides the octave. Each note is three half
steps from the previous and from the next. A distance in pitch of three half
steps is also known as a minor third, so fully diminished chords can be also
viewed as a series of minor thirds.

The structure, again, is

 d7
 d5
 m3
root

and the spelling of a C fully diminished seventh chord is C, Eb Gb and Bbb
(the \emph{diminished} seventh). C to Eb is a minor third, Eb to Gb is a minor
third, Gb to Bbb is a minor third and if we look at the Bbb as A for a moment,
we can see that A to C is a minor third.

C              Eb             Gb            Bbb             (C)
C    C\#   D    D\#   E    F    F\#   G    G\#   A    A\#   B    (C)

So: fully diminished chords equally divide the octave. This means that fully
diminished chords can have any of their chord tones (root, 3rd, 5th or 7th)
be the root. This means we have four fully diminished chords in one!! So:
C fully dim. 7 is the same as Eb fully dim. 7 is the same as Gb fully dim. 7
is the same as A fully dim. 7. They are the same because they have exactly
the same notes!

How many different fully diminished 7th chords are there? There are four
different notes, so 12 divided by four is three therefore there are three
different fully diminished seventh chords: on C, on C\# and on D.

Given this vagueness and ambiguity, why does it matter how we spell these
chords? Why not spell a C fully diminished seventh chord as C, Eb, Gb and A?
The answer is that there is the 'strictly legit' way of spelling in which the
A must be spelled as Bbb in the above example (after all, B anything is a
seventh above C, while A anything above C is a 6th), and there is the way of
spelling that involves options, an example of which would be re-spelling the
'legit' Bbb as an A.

Remember: the options are \emph{only} applied to fully diminished chords!

There are three spelling options for notes in fully diminished chords.

First, avoid double-flats. For example, Fbb should be spelled as Eb, and
Bbb should be spelled as A. A Gb fully diminished chord's legit spelling
is Gb, Bbb, Dbb and Fbb... very messy for the reader... With options, this
chord is spelled Gb, A, C and Eb. (this example will be used again; keep it
in mind...)

Secondly, Fb and Cb should be avoided. Respell Fb as E and Cb as B.

The last option is a bit more complex. As stated to me originally, "Avoid
black keys as FLATTED roots." There are some provisos to this one, however.
If you are wearing the arranger's hat, keep the root that's there. But if
you have a choice (say, if you're composing the tune), then prefer the
sharped enharmonic equivalent. Instead of writing Bb fully diminished (which
would be Bb, Db, Fb and Abb legit and Bb, Db, E and G using options), write
A\# fully diminished instead: A\#, C\#, E and G (notice we didn't need options
here). Instead of that \emph{particularly} messy Gb fully diminished from the
above example, write F\# fully diminished instead: F\#, A, C and Eb (again,
no options necessary).

To summarize the options, I say again that options should ONLY be used on
fully diminished chords. Options should NOT be used on chords that come
from a family grouping, such as I-major, II-major or V.

  o Avoid double flats:
      respell as simplest enharmonic equivalent
      i.e., Bbb should become A
  o Avoid Fb and Cb
      respell Fb as E
      respell Cb as B
  o In situations where you have creative choice,
    prefer a sharp as opposed to a flat in the name
    of a root on a black key.
      i.e., instead of a chord like Gb, Bbb, Dbb and Fbb,
      choose F\# for the root, as F\#, A, C and Eb.
    (Composers generally have creative choice; arrangers
    generally do not.)

No extensions??

That's right. Fully diminished chords have no 9ths, 11ths or 13ths. In the
example of a C fully diminished 7th chord (as in all fully diminished chords),
all the notes are a minor third apart. With this in mind, let's look for
extensions: C is the root; a minor third up is Eb, the third, a minor third
up from that is Gb, the fifth and a minor third up from that is Bbb, the
seventh. Applying options, we respell the Bbb as its enharmonic equivalent, A.
>From A, we go up a minor third to find... C. So is C the ninth? If it is, we
have an inconsistancy, because C is actually the root. Hence, there is no
ninth, and therefore also no eleventh or thirteenth.

So there's no such thing as a fully diminished 9th chord, no fully diminished
11th and no fully diminished 13th, because these notes are simply repeats of
the lower notes. However: there are other notes that work with the chord.

Any of the notes can be "leaned" on to introduce more tension, and "released"
to relieve that tension. In the case of the fully diminished chord, any note
can be leaned on by raising it a major second. The "leaned" note is released
by lowering it to its original, as in the following examples:

\lilypond[quote,fragment,staffsize=20]{<c' ees' ges' a'>}

------------------------------------------------------

------------------------------------------------------

------------------------------------------------------
                   7th      O
----------------5th-----b-O---------------------------

----------------3rd-----b-O---------------------------

               root  --  -O-
can become


------------------------------------------------------
                           |
---------------------------|----|---------------------
                           |    |
------------"leaner"------O-----|---------------------
                               O  - 7th (release or resolution)
---------------5th------b-O---------------------------

---------------3rd------b-O---------------------------

              root  --   -O-


In the first example, we see a C fully diminished chord. In the second, the
seventh is "leaned" on WHEN THE WHOLE CHORD IS STRUCK, and released while the
chord is still playing.

The seventh, A, is raised a major second to become B, which is struck with the
chord. This is the "leaning". Later, while the chord is sounded, the B is
brought back down to A. This is the "release". Get to a piano and play this!!

You could do the same thing to the fifth, you'd get Ab going to Gb. If you did
it to the third, you'd get F to Eb... You could even do it to the root! You
would get D going to C.

In some types of music, the leaned note is frozen, and does not release. In
others, you can lean on two notes (generally better the fifth and seventh) at
once, releasing them, or not...

The Fully Diminished Scale

The fully diminished scale consists of all the chord tones of the fully
diminished chord, plus all the leaner notes. Remember: each leaner note is
located a major second above any chord tone of a fully diminished chord.
So for example, the chord tones of C fully diminished 7th are C, Eb, Gb and A:

C              Eb             Gb             A              (C)
C    C\#   D    D\#   E    F    F\#   G    G\#   A    A\#   B    (C)

The leaner notes (I'll put them on the bottom) are:

C              Eb             Gb             A              (C)
C    C\#   D    D\#   E    F    F\#   G    G\#   A    A\#   B    (C)
          D              F              Ab             B

And: the leaner notes also form a fully diminished chord! (Were you surprised?)
Here they are on the same (top) side this time:

C         D    Eb        F    Gb        Ab   A         B    (C)
C    C\#   D    D\#   E    F    F\#   G    G\#   A    A\#   B    (C)

Notice the pattern of the scale: Whole step, half step, whole step, half step,
whole step, half step, whole step, half step. So we are alternating whole steps
with half steps.

Remember: to form a fully diminished scale that fits with a fully diminished
chord, start at the root and alternate upward, starting with a whole step (C
to D, in this case) and continuing half step, whole step, half, etc until you
get to the top.

Naming Fully Diminished Chords With "Leaner" Notes

As I said earlier, the fully diminished chords have no extensions, i.e., no
9ths, 11ths or 13ths. Given this, there must be a different mechanism for
naming fully diminished chords with these leaner notes. The standard way
is to use the word "add" followed simply by the letter name(s) of the leaner
note(s). For example, C, Eb, Gb, and B would be C fully dim. 7 add B.

Note: Just using monospaced ascii (as I'm forced to at the moment), there is
no way to really show the way the chord symbols are written. You do want to
see them. Suggestions? .gif? .ps? I'd like it to be something everyone can
see, so portability is important. It would make staff examples easier too.

If we were to alternate starting with a half step, we would actually get a
scale that would fit a dominant chord.

Do Fully Diminished Chords Exist?

There are two groups of theorists and composers on opposite sides of this
issue. One group thinks that the VII chord which can be either a fully- or
half-diminished chord, is truly a chord that stands on its own merrits and
has its own beauty.

(maybe add support for the pov described above here)

The other group feels that the fully diminished chord functioning as an
altered VII chord in a minor key, does not exist. These folks prefer to take
the root of that fully diminished chord down a major third to find its "true
root", on V, not VII. Overall, they feel that fully diminished chords exist
only in theory, and are really dominant chords. For more information on
finding the "true root" of a chord, research the harmonic overtone series.
A good treatice is in Gordon Delamont's "Modern Harmonic Technique, Vol. I",
pp. 25-36.

I use this point of view to show how the dominant chord is related. To
illustrate, we will take a B fully diminished 7th, spelled B, D, F and Ab. The
root here is B. We will add a note a major third below, giving G, B, D, F and
Ab. What kind of chord is this? The root is G, and recall that we look at the
third and seventh to determine the overall quality. G to B is a major third and
G to F is a minor seventh. Recall from the previous lesson that a major third
and minor seventh make a dominant chord. G to D is a perfect fifth, not much
surprise there. But G to Ab is not a major ninth, it's a minor ninth. That makes
the name of this chord, a G7(b9).

Looking at this situation in terms of the chromatic scale:

                  root           third          fifth         seventh
                    B              D              F              Ab
G    G\#   A    A\#   B    C    C\#   D    D\#   E    F    F\#   G    G\#

we now add a note a major third below the root, B to get G, which we are
calling the "true root".

Root              third          fifth         seventh        flat ninth
G                   B              D              F              Ab
G    G\#   A    A\#   B    C    C\#   D    D\#   E    F    F\#   G    G\#

Look now at the whole step-half step fully diminished scale on B:

B         C\#   D         E    F         G    Ab        Bb   B
B    C    C\#   D    D\#   E    F    F\#   G    G\#   A    A\#   B

Notice there is a G! This means the fully diminished scale built on the
original B fully diminished chord fits this new G7(b9) chord! How do we
know, theoretically speaking? Because all the notes of the chord are in the
scale! Observe:

third        fifth         seventh    root   b9           third
B         C\#   D         E    F         G    Ab        Bb   B
B    C    C\#   D    D\#   E    F    F\#   G    G\#   A    A\#   B

Another way of looking at this, is that the third, fifth, seventh and ninth
of a dominant chord with a flatted ninth form a fully diminished seventh
chord. One of the leaner notes is the root! Guess what, folks? Any one of the
leaner notes can be the root of a (different) dominant chord with a flat nine!

Let's look at the same B fully diminished scale, starting from the "true root"
of G:

G    Ab        Bb   B         C\#   D         E    F         G
G    G\#   A    A\#   B    C    C\#   D    D\#   E    F    F\#   G

Same fully diminished scale. All the same notes. This works with a B fully
diminished chord. This works also with a G7(b9) chord. Let's look closer at
the pattern: notice we are still alternating whole steps and half steps...
But: this time we are starting with a half step! So: we have found a scale
that works well with a dominant chord with a flatted ninth: the half step-
whole step fully diminished scale. We also know what works with a fully
diminished chord: the whole step-half step fully diminished scale.

So: this seemingly mild-mannered fully diminished scale will work with no less
than eight chords! I list them: B fully dim., D fully dim., F fully dim., and
Ab fully dim. Also: G7(b9), Bb7(b9), C\#7(b9) and E7(b9). That's alot of chords
to fit one scale, wouldn't you say?

Summary

We have covered four chordal groupings so far.

------------------------------------------------------
| -13  M13      |   M13    |  M13       |    !13     |
|      P11  A11 |   P11    |  P11  A11  |    !11     |
|  -9   M9   A9 |    M9    |   M9       |     !9     |
|       -7      |    -7    |   M7       |     d7     |
|       !6      |    !6    |   M6       |     !6     |
|  d5   P5   A5 |    P5    |   P5 (A5)  |     d5     |
|       P4      |    !4    |   P4       |     !4     |
|       M3      |    -3    |   M3       |     -3     |
|      Root     |   Root   |  Root      |    root    |
|---------------|----------|-------------------------|
|        V      | II-Major | I-Major    |  Bridging  |
------------------------------------------------------

Two left to go.

See ya next time...

jimlynch@netcom.com

P.S:

For the Italians of the group, you should know that the
Italian word "appogiare" (sp?) is, in English, "to lean",
and "appogiatura" is the past participle thereof.

In most modern theory books, these "leaner" notes are
referred to as appogiaturas; the appogiaturas discussed
here are, of course, those above the chord tones of a
fully diminished chord. I'm not sure if I have spelled
these correctly; someone correct me if I'm wrong.

