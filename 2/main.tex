\section{Lesson 2}

OK gang... This is the promised second chord study lesson.

As you recall, the last lesson concerned the minor 7 chords and their 
friends, minor 9 chords, minor 11 chords and minor 13 chords (rare).
The structure of these chords are:

        Major 13th
       Perfect 11th
        Major 9th
        Minor 7th
       Perfect 5th
        Minor 3rd
          Root

Note that all mention of intervals are intervals measured from the ROOT 
of the chord. The same is true for my first article.

Recall also that the goal of this series is to impart a knowledge of 
chords such as those you might see in a jazz fake book, for example

Dm11     G13     Cmaj.7

Specifically, the goal is to be able to spell a chord given its name and 
to give the name of the chord given its spelling. For 3 examples, the 
Dm11 above is spelled D, F, A, C, E and G; the G13 is G, B, D, F, A, C 
and E; and the C Maj. 7 is C, E, G and B.

Also, the chords covered here are ones that you could strike all the 
notes at the same time and hold them forever without being overly 
dissonant. Notes of chords which do not fall into this category would 
probably be considered non-harmonic or at least "un-hip".

(a quote from the first article:)

There are roughly 6 different chordal groupings you would need to worry about:
Tonic major chords, tonic minor, supertonic major, supertonic minor, dominant
and fully diminished. Each one of these has several possibilities for scales
to go with them (however, there are also non-scalar possibilities as well).

(unquote)

We will now look at tonic chords in major (all this means is I'm 
referring to chords whose ROOT is the FIRST (or tonic) degree of a MAJOR 
scale. For short, I will refer to chords in this family as I-Maj. chords. 

Since the chords dealt with in my last article had their ROOT be the SECOND 
(or supertonic) degree of a MAJOR scale, I referred to those (dorian) chords 
as II-Maj.

The first I-Maj. chord I discuss is a major triad with an added interval 
of a major 6th. In the key of C major, this would be (from the bottom) 
C, E, G and A; its name is C add 6 or just plain C6. The added 6th (\emph{always} 
a major 6th) is a \emph{substitute} for the interval of a major 7th, which leads 
to the second chord.

The next chord is a major triad with a major 7th interval. An example 
would be C Maj. 7 which would be spelled C, E, G and B. As discussed in 
the preceding paragraph, the add 6 chord is a substitute for the maj. 7 
chord because the major 6th interval is a substitute for the major 7th 
interval. In the '40s, the add 6 chord was used exclusively to harmonize 
melodies at points of tonic major harmony. Presently (big band), major 7 
and add 6 are used interchangably. 

Next is a major triad with an added 6th and major 9th. In the key of C, 
this would be C, E, G, A and D. Its name would be C 6/9 which is 
pronounced "Cee six nine". It substitutes for the following chord.

(Note that all chords in the same family can substitute for each other, 
e.g. if the fake book says C Maj. 7, could you substitute a C 6/9? Yes! 
How about a C Maj. 9? Absolutely! Or a C6? Definitely! Any other chord in 
the I-Maj. category would make a good substitute.)

Next is a major triad with the intervals of a major 7th and major 9th. 
The example would be (in the key of C major) C, E, G, B and D. C Maj. 9 
is its name. Its (main) substitute is a C 6/9 due to the use of the
substitute for the interval of a major 7th, which is a major 6th.

Next is a major 11 chord which, in the key of C major, would be spelled 
C, E, G, B, D and F. This chord has the same structure as a major 9th 
chord with an added perfect 11th. The name of this chord would be C Maj. 11. 

This chord requires some discussion: the third of any chord is very 
important since it determines whether the triad is major or minor. 
If you place a note in the chord which is overly dissonant with that 
third, you interfere with its "major vs. minor identification" function. 
The perfect 11th, being exactly an octave larger than a perfect fourth, 
is such a dissonance. It creates the interval of a minor 9th with the 
third, which is a very sharp dissonance.

For the reasons stated above, the next chord is almost always used instead of 
the preceding. In the key of C major, it would be C, E, G, B, D and F\# (note 
\emph{lydian} influence) and would be called C Maj 9 (+11) or C Maj 9 (\#11).
The structure is that of a Maj. 9 chord with an added augmented 11th.
Here, the interval created between the third and the 11th of the chord is 
a \emph{major} 9th, a mild dissonance, and the problems discussed above are no 
longer present. Note the syntax used: the altered interval(s) are in 
parentheses and the number outside the parens is the highest \emph{unaltered} 
interval \emph{for that chord family.} Since a '9' is outside the parens, the 
interval of the 9th in the chord is its highest unaltered interval.

Note carefully here that this means there are two versions of the 11th! 
In this case, the \emph{altered} (raised) 11th is preferred. However: you have 
to know that the unaltered 11th is a perfect 11th (even tho it's rarely 
used and even undesirable for the reasons discussed above.)

The next chord is called a major 13 chord and its structure is that of a 
maj. 11 chord with the interval of a major 13th added on top. In the key
of C major, that would be C, E, G, B, D, F and A. The problem described 
in the description of the major 11 chord (perfect 11th produces too much 
dissonance with the major third) also exists here. For this reason...

The next chord's type is a major 13 with a raised 11th. This alleviates 
the dissonance problem between the third and 11th. Its structure is that 
of a major 9 w/ raised 11th with an added major thirteenth. In the key of 
C major, this is C, E, G, B, D, F\# and A.

The last two chords have a frozen suspended fourth in them. In Bach's 
time, a "sus 4" is a note in a chord located a perfect fourth above its
root, and this "suspension" note would "resolve" to the third of the 
chord within the same harmony. There are other requirements for this 
(melodic) device, but to describe them all would unnecessarily bog us 
down. I refer those interested to Gordon Delamont's "Modern Harmonic 
Technique Vol. 2", chapter 3. The term "frozen suspension" simply means
that the suspension note remains in the chord and does not resolve.

The first of these is a major 7 sus 4, it has a root, \emph{no} third, perfect 
4th and major 7th. In the key of C major: C, F, G and B. Its name (for the
key of C major) is C Maj. 7 (sus 4). Note that the "sus" can be notated 
four ways in the alphabetical chord system: sus, (sus), sus 4, (sus 4), 
so other names for this chord include C Maj. 7 (sus), C Maj. 7 sus, and 
C Maj. 7 sus 4. So, this chord could be notated in three other ways:
C Maj. 7 (sus), C Maj. 7 sus 4, and finally C Maj. 7 sus. 

The last chord in this family is a Major 9 with a sus 4. The structure 
is that of a major 7 sus 4 with an added major 9th. In the key of C major, 
that would be C, F, G, B and D.

Look carefully at the composite intervalic structure for I-Major chords:

        Major 13th
       Perfect 11th   Augmented 11th
        Major 9th
        Major 7th
        Major 6th
       Perfect 5th
       Perfect 4th
        Major 3rd
          Root

Did you notice that the augmented 11th (an alteration!) is not stacked 
above the root? This means that in the alphabetic chord symbol, an 
augmented 11th would be shown as an alteration: In Parentheses. Example:
C Maj. 9 (\# 11).

If we "scalize" the structure (look at the 9th, 11th and 13th as the 2nd, 
4th and 6th, we can come up with at least two different scales for 
improvising on this chord type:

Root, Maj, 9th, Maj, 3rd, Prf. 4th (the sus!), Prf. or Aug. 11th, 
Prf. 5th, Maj. 6th (or 13th), Maj. 7th, Root.

So it looks like the two scales that would work are ionian and lydian. 
Note that with ionian, you must use its perfect 4th with care as it shows 
a strong tendency to lead to the (major) 3rd.

Compare this with the structure of the II-Major family:

\begin{tabular}{ | l c l | }
	\hline
	Maj 13th					&					& Maj 13th				\\
	P 11th   A 11th   & 				& P 11th					\\
	Maj 9th           & 				& Maj 9th					\\
	Maj 7th    				& $\iff$	& Min 7th					\\
	Maj 6th						& $\iff$	& NO 6th!					\\
	P 5th             & 				& P 5th						\\
	P 4th							& $\iff$	& NO 4th!					\\
	Maj 3rd						& $\iff$	& Min 3rd					\\
	Root              & 				& Root						\\
	\hline
  I-Maj.            & 				& II-Maj.					\\
  \hline
\end{tabular}

"Scalizations":

I-Maj:

Root, Maj, 9th, Maj, 3rd, Prf. 4th (the sus!), Prf. or Aug. 11th, 
Prf. 5th, Maj. 6th (or 13th), Maj. 7th, Root: Lydian mode or Ionian mode.

II-Maj:

Root, Major 9th, Minor 3rd, Perfect 11th, Perfect 5th, 
Major 13th, Minor 7th, Root: this scale is the Dorian mode.

