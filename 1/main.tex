\section{Lesson 1}
In article \textless93040.150046DAHUGH@LIVERPOOL.AC.UK\textgreater, DAHUGH@LIVERPOOL.AC.UK writes:
\begin{quote}
  Could anyone shed some light on how to decide which mode to play over a chord progression?I've heard of Satch,Vai etc. talking about playing in a lydian tonality.What does this mean?How would you know to play for example D dorian or G Mixolydian when they contain the same notes? I'm lost.
\end{quote}

The answer to your question has many sides; I will make suggestions for 
further study later; meanwhile, I go directly to your request: shed \emph{some} 
light.

First, given a chord progression, the scales you would play over that
progression come from the chords, not the chords from the scales.

(Note, I have been evaluating new information wrt this point, and my
conclusion now, is: chronologically/historically, melodies and scales
came first, then chords were built from the scales. However, from a
-physical- standpoint, chords and scales -both- come from the harmonic
overtone series, which provides information such as 'the safe low limit
for a particular chord voicing'. More later.)

There are roughly 6 different chordal groupings you would need to worry about:
Tonic major chords, tonic minor, supertonic major, supertonic minor, dominant
and fully diminished. Each one of these has several possibilities for scales
to go with them (however, there are also non-scalar possibilities as well).
The first five groups mentioned can be said to 'come from a key' while the last
(fully diminished) has no allegiance to any key (due to the fact that this
chord's four notes equally divides the octave and therefore has four possible
roots). So I will refer to the first five as chord 'families' and the last
as just a chord 'group'.

For this example I take the supertonic major family because of its simplicity.
The word 'tonic' as used here refers to the first degree of the major or minor
scale and 'supertonic' refers to the second. So, 'supertonic major chord family'
simply refers to all (good-sounding) chords whose root is the second degree of
a major scale. As you may recall, the Dorian mode is also built on the second
degree of the major scale, and as you will see, the Dorian mode is a good
scale to use on a II chord that comes from a major key.

I'm sure you know about triads and just a bit less sure that you're aware of
chords with 7ths, but did you know that chords can have extensions? The 9th,
11th and 13th are the extensions on \emph{family} chords (fully dim has no
extensions, but... more later).

The supertonic major chord family (or just II chords coming from major -- same
thing) have the following intervalic structure:

        Major 13th
        Perfect 11th
        Major 9th
        Minor 7th
        Perfect 5th
        Minor 3rd
        Root

altho not all these notes need be present. If you bring the 9th, 11th and 13th
down an octave, you get a (Major) 2nd, (perfect) 4th and (major) 6th, and now
we can look at this chord as a scale:

Root, Major 2nd, Minor 3rd, Perfect 4th, Perfect 5th, Major 6th, Minor 7th, Root

which is the same as a Dorian scale. So: if you are given a Dm7 (II of C),
you could play a D Dorian (or a C Ionian or G Mixolydian...) scale over that
chord and it would sound fine.

I mentioned picking this family for its simplicity. It's so simple for three
reasons: 1) there are no alterations (i.e., no flat nines or sharp fives, etc);
2) all notes come from the key; and 3) there are no sus 4ths nor added 6ths,
the chord is just straight up root, 3, 5, 7, 9, 11 and 13. I have not discussed
the chord families that have more than one possibility for certain intervals,
for example the Dominant (built on the 5th degree of a major or minor scale),
which can have a perfect fifth as well as a raised fifth and a lowered fifth.
In the case where alterations are possible, a pure scale approach is less
possible or desired.

I also haven't discussed soloing over the change of harmony, which is beyond
the scope of this article. For this, I suggest taking a harmony course since
while I can talk about chords in about 10-20 minutes, voice-leading over a
change takes about a year. This class will necessarily involve reading and
writing manuscript; in addition, I highly recommend you apply the concepts
to your instrument at every step. (I still have yet to do this... )-: )

If there's enough interest (as evidenced by email in my box), I will go through
the other chordal groupings. Flames to /dev/null. Similarity or lack thereof
to anyone's theory or opinion is purely concidental and not my fault.

With this backround, I now turn to Kenny A. Chaffin's question. While at least
part of the answer is implied already, I make it explicit here.

Kenny,

You need the modal \emph{fingering patterns} so you can play any scale no matter
where your left hand is. Modal \emph{compositional approach} has to do with the
fact that each mode has a different flavor, and as I implied above, can be
(at least sometimes) derived from the chord that's happening at the time.

Practice two-octave scales for each mode in one key all the way up and down
the neck. Do this with up-down picking. Repeat with down-up picking.

If you're really interested in modes, repeat previous paragraph with
a tone generator sounding mode root, for example in C major: repeat with
synth playing C, repeat with synth playing D, E, F, G, A and B.

Repeat previous two paragraphs for all keys in this order: F C G D A
E B Gb Db Ab Eb Bb F. You will notice with each key change that only one note
changes, and only a half step.

When you get \emph{really fast}, you will be able to do all this once a day.
This constitutes modal \emph{fingering} and modal \emph{ear training} combined.
You will probably need 6 months to a year working on this stuff; use
a metronome and practice slowly and gradually increase over a period
of at least 3 months. Be \emph{carefull}, otherwise you'll get tendonitis
like Yngwie. \emph{Always} warm up, especially when you develop any kind of
speed.

If you do this, you will gain the ability to start in playing and knowing
where you are on your fingerboard immediately no matter where your hand
was before, including completely off the guitar. If not, you won't...

Of course, this is another area that I have yet to accomplish... So
you can start by just listening to the different qualities of the
modes.

Apparently, there's some interest in chordal theory. Give me some time and
in the next few days I'll start talking about the versatile dominant structure.

For those who want something to do, play and write these chords. Suggestion:
write Xmi7, Xmi9, Xmi11 and Xmi13, where X is the second degree in each of
fifteen major key signatures (so that's 15 times four = sixty chords.)

Write straight up, i.e., F\#mi7 (from the key of E major, 4 sharps) would be
F\#, A, C\#, E; F\#mi9 is F\#, A, C\#, E, G\#, etc. For those interested in
manuscript, a knowledge of key signatures is a must as well as knowing how
to identify intervals on the staff. We will be developing the ability to
spell the notes of a chord given its name and give the name of a chord when
provided with its spelling. When given the spelling, identify the intervals
from the root of the chord. Example: F\#, A, C\#, E, G\# has a minor 3rd (F\#-A),
a perfect fifth (F\#-C\#), a minor 7th (F\#-E) and a major 9th (F\#-G\#).

Here's a little quiz:

Spell and identify the major key:

Cm13   Dbm11   G-9   Fmi7   Abm11

Identify the name of the chord and the key:

Eb Gb Bb Db F Ab       A C E G B D F\#      Bb Db F Ab C Eb G

B D F\# A C\#          E G B D F\# A C\#

Also identify all the intervals with respect to the root.

Notice that these chords are built in thirds -- hardly playable on a guitar.
Why not find your own voicings? Remember that you don't need all the notes -
leave some notes for the horns (or lead guitar or whatever) and the root
need not be on the bottom (or even be present as long as someone's playing
it). Use your ear! Find good-sounding voicings and write, post and use them!
