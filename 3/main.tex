\section{Lesson 3 - Dominant Chord Structures}

Yet another chord structure is explored in this, my third chord lesson.
So far, we have examined two chord structures, namely tonic (I) chords
that come from a major key (I-maj. for short), and supertonic (II) 
chords that come from a major key (II-Maj. for short).

A comparison of the structures of these chords by intervalic chordal member:

\begin{tabular}{ | l c l | }
	\hline
	Maj 13th					&					& Maj 13th				\\
	P 11th   A 11th   & 				& P 11th					\\
	Maj 9th           & 				& Maj 9th					\\
	Maj 7th    				& $\iff$	& Min 7th					\\
	Maj 6th						& $\iff$	& NO 6th!					\\
	P 5th             & 				& P 5th						\\
	P 4th							& $\iff$	& NO 4th!					\\
	Maj 3rd						& $\iff$	& Min 3rd					\\
	Root              & 				& Root						\\
	\hline
  I-Maj.            & 				& II-Maj.					\\
  \hline
\end{tabular}

To keep us on track, here are relevent quotes from previous articles:

Quote from second article:

Note that all mention of intervals are intervals measured from the ROOT
of the chord. The same is true for my first article.

Recall also that the goal of this series is to impart a knowledge of
chords such as those you might see in a jazz fake book, for example

Dm11     G13     Cmaj.7

Specifically, the goal is to be able to spell a chord given its name and
to give the name of the chord given its spelling. For 3 examples, the
Dm11 above is spelled D, F, A, C, E and G; the G13 is G, B, D, F, A, C
and E; and the C Maj. 7 is C, E, G and B.

Also, the chords covered here are ones that you could strike all the
notes at the same time and hold them forever without being overly
dissonant. Notes of chords which do not fall into this category would
probably be considered non-harmonic or at least "un-hip".

(unquote)

From the first article:

There are roughly 6 different chordal groupings you would need to worry about:
Tonic major chords, tonic minor, supertonic major, supertonic minor, dominant
and fully diminished. Each one of these has several possibilities for scales
to go with them (however, there are also non-scalar possibilities as well).

(unquote)

With old business out of the way, let's examine the dominant structure.

The dominant chord structure is the most flexible structure we will look at.
A light root-3rd-7th voicing can be made dark and rich by the addition of
altered fifths and ninths. The darkness can be replaced by richness alone by
using the unaltered versions. In fact, we will see that the entire chromatic
scale less two notes can be used in solos.

Dominant chords are those whose ROOT is the FIFTH (or dominant) degree of
a major scale. We can get dominant chords from the fifth of melodic and
harmonic minor, as well as certain other modes of these scales. For short,
I will refer to chords in this category as being in the V family.

As far as naming goes, recall that a "7" in an alphabetical chord symbol
makes reference to the interval of a minor seventh above the root. We saw
in the first lesson how this was used: Dm signified a D minor triad, Dm7
would add the interval of a minor seventh to that minor triad. So the
symbol Dm7 seen in a fake book would be broken down like: 

          Dm                 7
      The TRIAD       the MINOR seventh.

Dominant chords have a MAJOR third and a minor seventh, so one possible
chord would be G7, the interpretation being:

         G                    7
    The TRIAD          The MINOR seventh.

Compare this to I chords in major: CMaj7 would be interpreted:

        C                Maj7
   The TRIAD         THE !!MAJOR!! SEVENTH!

Check this out! The m in Dm7 refers to the triad, while the Maj in CMaj7
refers to the 7th! Don't let this confuse you: The triad is \emph{already} 
major, so it wouldn't do to say C Maj Maj 7! (meaning C Maj triad with
a Maj 7th) Remember the triad naming rules: the note name by itself always
means a MAJOR triad, i.e., C means C major triad; G means G major triad.

Observe the way the name is constructed: G7 (note name   number) versus
GMaj7 (note name  modifier  number) versus Dm7 (note name  modifier  number).

So dominant chords are generally named with a note name followed by a number.
Let's look at the structure so far: a major triad with a minor seventh looks
like:

        Min 7th
         P 5th
        Maj 3rd
         Root

          V

Play the chord. Hear the sound. Chords like this: G7, Eb7, B7, etc. Remember,
when you look at the structure, the intervals directly above the root (the
only ones so far) are the unaltered intervals for V family chords.

The fifth of the dominant can be raised or lowered one half step. If this is
done, the altered fifth would be shown in parentheses: G7(b5) or G7(\#5) or
G7(-5) (this is the same as G7(b5)) or G7(+5) (same as G7(\#5)). If you took
the 7 away, you'd have a major triad with a raised or a lowered fifth.

Our new structure looks like:

        Mi 7
 d 5th   P 5   A 5
        Ma 3
        Root

         V

Because I would like to eventually put all six chord family structures
together in order to compare them, I would like to reduce their horizontal
size. Take a look at this structure, \emph{identical} but reduced:

       -7
  d5   P5   A5
       M3
      Root

        V

PLAY these chords! Hear the sounds of the altered fifths! Get to a piano! Your
ear is too important to ignore! Some altered-fifth chords: G7(\#5) F7(-5) etc

Abbreviations used: 
abbrv.  meaning           examples
------  -------           --------
  M      major         M3 (major third)
  -      minor         -7 (minor seventh)
  P      perfect       P5 (perfect fifth)
  A     augmented     A11 (augmented eleventh)
  d     diminished     d5 (diminished fifth)
  !     not present    !6 (sixths are not present
        in structure       in the structure being
                           discussed)

Because the diminished 5th and the augmented fifth are seen in the structure as
to one side of the root, as opposed to directly above the root, it would be 
shown as an alteration (in parentheses). So that means the maj 3rd, the p 5th
and the min 7th are unaltered intervals WITH RESPECT TO the dominant chord.
Other chord structures have \emph{different} unaltered intervals! And: that's how
we can place them in their category. Compare it, for example, to the II-Maj
structure, up to the seventh:

-------------------------------
|    -7     |       -7        |
|    P5     |  d5   P5   A5   |
|    -3     |       M3        |
|   Root    |      Root       |
|-----------|-----------------|
| II-Major  |        V        |
-------------------------------

Fron this chart, we can see that the intervals of the seventh in both
structures is minor, but the thirds differ. In II-Major, the third is
minor, but in V, the third is major, and in each the third is unaltered
FOR THAT STRUCTURE.

Another way to use this comparison is for figuring out to which structure
a given chord belongs, for example we can see that if we find a major
third in the chord, it could not possibly be in II-Major. What if the
chord is found to have the interval of a major 7th? Could it possibly
be a dominant chord? More on this later.

Here's an important piece of trivia: The dominant chord structure is the
\emph{only} group of chords that can tolerate altered ninths. In all other
families, the ninth is a major ninth. The dominant, on the other hand,
has the minor ninth and the augmented ninth as possibilities. Here's the
updated structure:

-------------------
| -9    M9   A9   |
|       -7        |
|  d5   P5   A5   |
|       M3        |
|      Root       |
|-----------------|
|        V        |
-------------------

So now there are at least nine possibilities, among them G9(+5), F\#7(b5,\#9),
Db7(\#9), etc.

Play all the possibilities. Bass players arpeggiate them or play them at high
positions on the neck. Guitar players can just play the chords, but playing
them as arpeggios is also valuable. To gain facility, take a specific chord
and play it everywhere on the neck in all keys. With the possibilities that
we have now, this will take you quite awhile.

Dominant chord structures can have two kinds of elevenths: the perfect
eleventh and the augmented 11th (note lydian influence). As in I-Major,
the perfect 11th is problematic.

For more on this, I quote from the second lesson:

This chord requires some discussion: the third of any chord is very
important since it determines whether the triad is major or minor.
If you place a note in the chord which is overly dissonant with that
third, you interfere with its "major vs. minor identification" function.
The perfect 11th, being exactly an octave larger than a perfect fourth,
is such a dissonance. It creates the interval of a minor 9th with the
third, which is a very sharp dissonance.

(unquote)

If you see a chord such as G11 (normally G, B, D, F, A and C) in a fake book,
you will want to drop the third (G,  , D, F, A and C, note missing note). Note 
that here, you have a G as the root and D, F, A and C on top. The upper notes 
can be viewed (and written as) a Dm7. So you could write this chord as:

                Dm7 /
                   /
                  /  G

which means Dm7 on top, put a G in the bass. !Note! that this is different from

              Dm7
              ---
               G

because this means play the two \emph{chords} Dm7 on top of a G major triad.

The perfect 11th can also be viewed as a sus 4. For example, the perfect 11th
above G is C, which is also a perfect 4th. Recall from lesson 2 that sus 4
chords have no third.

Dominant chords can also have raised 11ths, in which case the third of the
chord can comfortably exist. This would introduce a "lydian" flavor to the
chord, which would otherwise be a "mixolydian" flavor. This is because it
would have the "mixolydian" interval of the minor seventh and the "lydian"
raised 11th. Otherwise unaltered chords of this variety are called, therefore,
"lydian dominant" chords, such as G9(+11), E7(\#11), etc.

The structure so far:
-------------------
|      P11  A11   |
| -9    M9   A9   |
|       -7        |
|  d5   P5   A5   |
|       P4        |
|       M3        |
|      Root       |
|-----------------|
|        V        |
-------------------

Recall from previous lessons what the number outside any parentheses means:
that number says that the HIGHEST \emph{unaltered} interval is that number. The
6th does not exist in dominant harmony, but the 13th does. What does this
mean? The 13th IS the 6th! In dominant harmony, the interval of the minor
seventh is always present in the chord, so a G add 6 (or G6) is \emph{not} dominant
harmony. This is because the chord symbol says that the 6th is the highest
unaltered interval, which conflicts with my statement that dominant chords
always have sevenths. Recall that an added 6th (always a major 6th) is a
substitute for a \emph{major} seventh. Given this, our G add 6 chord, with its
major third and major seventh substitute, is I-Major harmony. (see lesson
2 for details)

However, dominant chords \emph{can} have 13ths, because the 13 in a chord symbol
like G13 implies the existance of at least the major 9th and minor 7th.
The major 13th is the norm. Although most of the time the minor 13th should
be looked at as an augmented fifth (that way, you don't go adding more than
one fifth), the minor 13th is also possible.

Here is the full dominant structure:

-------------------
| -13  M13        |
|      P11  A11   |
| -9    M9   A9   |
|       -7        |
|       !6        |
|  d5   P5   A5   |
|       P4        |
|       M3        |
|      Root       |
|-----------------|
|        V        |
-------------------

PLAY THESE CHORDS! Some will be ugly at first; you'll get used to them. Once
you've exausted the possibilities, start deciding which ones you like. That's
how to build your chordal vocabulary.

Some dominant chords: G7(b5), F13(+11), Eb9(\#5,\#11,b13)... the list goes on and
on.

Let's take all the structures we have so far and put them side-by-side:

-----------------------------------------
| -13  M13      |   M13    |  M13       |
|      P11  A11 |   P11    |  P11  A11  |
|  -9   M9   A9 |    M9    |   M9       |
|       -7      |    -7    |   M7       |
|       !6      |    !6    |   M6       |
|  d5   P5   A5 |    P5    |   P5 (A5)  |
|       P4      |    !4    |   P4       |
|       M3      |    -3    |   M3       |
|      Root     |   Root   |  Root      |
|---------------|----------|-------------
|        V      | II-Major | I-Major    |
-----------------------------------------

Notice that THE EXTENSIONS (9ths, 11ths, 13ths) DON'T TELL YOU MUCH in the way
of comparison! In fact, the unaltered extensions are !always! \emph{major} ninth,
\emph{perfect} eleventh and \emph{major} thirteenth. So how do we compare chords? The
fifths don't provide much information either, and the roots provide exactly no
information. 

What's left? The third, the seventh and their substutes. That's all! By
determining the types of the third and the seventh, you can immediately
place the chord in one of the categories so far, and we will see in the
next three lessons that the third and the seventh (or substitutes
thereof) will always categorize any chord into one of six chord
categories, either one of five families, or the fully diminished chord
group.  

What are the category names of the following chords?
C7(+5) F\#m7 D\# Maj. 9 (\#11) Ab9 Gm9 F6 Db6/9 Eb13 sus
What chord members do they have (list all the intervals in the chords)?
What are their exact spellings?

Measure the intervals of all the following spellings, and categorize each.
Then determine their exact name.
E G\# B D F\# A C\#
G B D\# F\# A C\#
D F\# A C Eb G\# Bb
B E F\# A C\#
F\# A C\# E G\#
Bb Db F Ab C Eb G
Gb Bb D Fb Abb
Cb Eb Gb Bbb Db F Ab
Eb G Bb C F
C F G Bb D
F Bb C Eb G
A C\# E G\# B\# D\#
D F A C E G B

"Scalizations":

V:

Root, Min. 9th, Maj. 9th, Aug. 9th, Maj. 3rd, Prf. 4th (the sus!), 
Prf. or Aug. 11th, dim. 5th, Prf. 5th, Aug. 5th, Min. 13th, Maj. 13th,
Min. 7th, Root.
Note: perfect 4th is same note as perfect 11th, except is 1 oct. lower
      enharmonic equivalents: dim. 5th and Aug 11th, 
                              Aug. 5th and Min. 13th
Disregarding the troublesome perfect 11th and the sus 4, this is the
entire chromatic scale except for the perfect fourth and the 
major 7th. Many scales are obviously possible, the most basic being 
mixolydian, another often-used one being mixolydian with a raised 4th.
The "blues" scale will also work. This is a minor pentatonic scale with 
an added chromatic tone between the 4th and 5th.

I-Maj:

Root, Maj. 9th, Maj. 3rd, Prf. 4th (the sus!), Prf. or Aug. 11th,
Prf. 5th, Maj. 6th (or 13th), Maj. 7th, Root:
Lydian (if 3rd present) or Ionian (if sus-4 alignment)

II-Maj:

Root, Major 2nd, Minor 3rd, Perfect 4th, Perfect 5th,
Major 6th, Minor 7th, Root:
Dorian mode.

References

If you don't know how to identify intervals, modes, key signatures,
rhythms, major and minor scales, chromatic scales or you want to know
how to write any of the above, see E. D'Amante's "Fundamentals of Music".
One can get info., from Ardsley House Publ. Co. at 320 Central Park West;
New York, NY 10025.

Most of the chordal theory you have seen and will see here is described 
in E. D'Amante's "All About Chords". In particular, the chordal groupings
are taken from D'Amante's Music 2A class, which begins with triads, goes
all the way to the 13th and contains several sections on chordal function,
which is beyond the scope of these articles. If you are interested in
receiving information about All About Chords, send a post card requesting
information from Encore Music Publ. Co. at P.O. Box 315 Orinda, CA 94563.

D'Amante was my teacher; he was the best theory teacher I have had to
date by several levels of magnitude. I used to say, "He scrambles my
brains every day and I love it." He is completing work on an ear training
text; it should be ready shortly.

While this theory stuff is interesting and can be useful for things like
figuring out what was in a composer's mind when he wrote the piece you're
examining, or writing your own tunes, nothing substitutes for your ear.
Play everything you write! Play every chord in all these lessons! (damn,
that's a lot of chords...) Experiment with all three chord structures!
Find all the ugly sounds! Find all the light sounds, and the heavy ones!
Find all the wondrous sounds! Find all the strange-but-beautiful sounds!
Build your vocabulary, then decide what you like.

See you next time.

